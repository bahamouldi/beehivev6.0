% ============================================================================
%  BeeWAF Enterprise v6.0 — Cahier de Charge Complet
%  Projet de Fin d'Études (PFE)
%  Février 2026
% ============================================================================
\documentclass[12pt,a4paper,french]{report}

% ----- Encodage et langue -----
\usepackage[utf8]{inputenc}
\usepackage[T1]{fontenc}
\usepackage[french]{babel}

% ----- Mise en page -----
\usepackage[top=2.5cm, bottom=2.5cm, left=2.5cm, right=2.5cm]{geometry}
\usepackage{setspace}
\onehalfspacing

% ----- Polices -----
\usepackage{lmodern}
\usepackage{microtype}

% ----- Couleurs et graphiques -----
\usepackage[table,xcdraw,dvipsnames]{xcolor}
\usepackage{graphicx}
\usepackage{tikz}
\usetikzlibrary{shapes.geometric, arrows.meta, positioning, calc, fit}

% ----- Tableaux -----
\usepackage{booktabs}
\usepackage{longtable}
\usepackage{tabularx}
\usepackage{multirow}
\usepackage{array}
\newcolumntype{L}[1]{>{\raggedright\arraybackslash}p{#1}}
\newcolumntype{C}[1]{>{\centering\arraybackslash}p{#1}}
\newcolumntype{R}[1]{>{\raggedleft\arraybackslash}p{#1}}

% ----- Code source -----
\usepackage{listings}
\usepackage{fancyvrb}

% ----- Hyperliens -----
\usepackage[hidelinks,colorlinks=true,linkcolor=NavyBlue,urlcolor=RoyalBlue,citecolor=ForestGreen]{hyperref}
\usepackage{bookmark}

% ----- En-têtes et pieds de page -----
\usepackage{fancyhdr}
\pagestyle{fancy}
\fancyhf{}
\fancyhead[L]{\small\leftmark}
\fancyhead[R]{\small BeeWAF Enterprise v6.0}
\fancyfoot[C]{\thepage}
\renewcommand{\headrulewidth}{0.4pt}
\renewcommand{\footrulewidth}{0.2pt}

% ----- Listes -----
\usepackage{enumitem}

% ----- Divers -----
\usepackage{caption}
\usepackage{subcaption}
\usepackage{float}
\usepackage{pdfpages}
\usepackage{tcolorbox}
\tcbuselibrary{skins,breakable}

% ----- Couleurs personnalisées -----
\definecolor{beeyellow}{HTML}{F5A623}
\definecolor{beedark}{HTML}{2C3E50}
\definecolor{beeblue}{HTML}{2980B9}
\definecolor{beegreen}{HTML}{27AE60}
\definecolor{beered}{HTML}{E74C3C}
\definecolor{codebg}{HTML}{F8F8F8}
\definecolor{codeframe}{HTML}{DDDDDD}

% ----- Style listings -----
\lstset{
  backgroundcolor=\color{codebg},
  frame=single,
  rulecolor=\color{codeframe},
  basicstyle=\ttfamily\small,
  keywordstyle=\color{beeblue}\bfseries,
  commentstyle=\color{beegreen}\itshape,
  stringstyle=\color{beered},
  numbers=left,
  numberstyle=\tiny\color{gray},
  numbersep=8pt,
  breaklines=true,
  breakatwhitespace=false,
  showstringspaces=false,
  tabsize=2,
  captionpos=b,
  xleftmargin=1em,
  framexleftmargin=1em,
}

\lstdefinelanguage{yaml}{
  keywords={true,false,null,y,n},
  sensitive=false,
  comment=[l]{\#},
  morestring=[b]',
  morestring=[b]",
}

\lstdefinelanguage{nginx}{
  keywords={server,listen,location,proxy_pass,upstream,return,ssl_certificate,ssl_protocols,add_header,proxy_set_header,worker_processes,worker_connections},
  sensitive=false,
  comment=[l]{\#},
  morestring=[b]',
  morestring=[b]",
}

% ----- Boîtes colorées -----
\newtcolorbox{infobox}[1][]{
  colback=beeblue!5,
  colframe=beeblue!80,
  fonttitle=\bfseries,
  title={#1},
  breakable,
  sharp corners,
}

\newtcolorbox{warnbox}[1][]{
  colback=beeyellow!10,
  colframe=beeyellow!80!black,
  fonttitle=\bfseries,
  title={#1},
  breakable,
  sharp corners,
}

\newtcolorbox{resultbox}[1][]{
  colback=beegreen!5,
  colframe=beegreen!80,
  fonttitle=\bfseries,
  title={#1},
  breakable,
  sharp corners,
}

% ============================================================================
\begin{document}

% ----- PAGE DE GARDE -----
\begin{titlepage}
\begin{center}

\vspace*{1cm}

{\Huge\bfseries\color{beedark} BeeWAF Enterprise v6.0}

\vspace{0.5cm}

{\LARGE\color{beeblue} Cahier de Charge Complet}

\vspace{1.5cm}

\begin{tikzpicture}
  \node[draw=beeyellow, line width=3pt, rounded corners=15pt, inner sep=20pt, fill=beeyellow!10] {
    \begin{minipage}{0.7\textwidth}
    \centering
    {\Large\bfseries Web Application Firewall Intelligent}\\[0.5em]
    {\large 10\,041 Règles $\bullet$ 3 Modèles ML $\bullet$ 27 Modules}\\[0.3em]
    {\large 7 Frameworks de Conformité}\\[0.3em]
    {\large\color{beegreen}\bfseries Grade : A+ (100\% Tests)}
    \end{minipage}
  };
\end{tikzpicture}

\vspace{2cm}

\begin{tabular}{rl}
\textbf{Projet} & BeeWAF --- Web Application Firewall Intelligent \\
\textbf{Version} & 6.0 (Février 2026) \\
\textbf{Auteur} & Équipe BeeHive PFE \\
\textbf{Classification} & Document Technique Complet \\
\textbf{Dernière mise à jour} & 10 Février 2026 \\
\end{tabular}

\vfill

{\small\color{gray} Document généré le 10 Février 2026\\
BeeWAF Enterprise v6.0 --- Grade Fonctionnel : A+ (260/260 tests, 100\%)}

\end{center}
\end{titlepage}

% ----- TABLE DES MATIÈRES -----
\tableofcontents
\newpage

% ----- LISTE DES TABLEAUX -----
\listoftables
\newpage

% ============================================================================
% CHAPITRE 1 : INTRODUCTION & CONTEXTE
% ============================================================================
\chapter{Introduction \& Contexte}

\section{Présentation du Projet}

BeeWAF Enterprise est un \textbf{Web Application Firewall (WAF)} de nouvelle génération conçu pour fournir une protection de niveau entreprise contre les attaques web. Développé dans le cadre d'un Projet de Fin d'Études (PFE), il combine :

\begin{itemize}[leftmargin=2em]
  \item \textbf{Détection par règles regex} : 10\,041 patterns compilés couvrant 50+ catégories d'attaques
  \item \textbf{Intelligence Artificielle} : Ensemble de 3 modèles ML (IsolationForest + RandomForest + GradientBoosting)
  \item \textbf{27 modules de sécurité} spécialisés couvrant tous les vecteurs d'attaque modernes
  \item \textbf{Conformité} à 7 frameworks de sécurité (OWASP, PCI DSS, GDPR, SOC2, NIST, ISO 27001, HIPAA)
\end{itemize}

\section{Positionnement}

BeeWAF surpasse les solutions commerciales de référence :

\begin{table}[H]
\centering
\caption{Comparaison BeeWAF vs solutions commerciales}
\label{tab:positionnement}
\begin{tabular}{@{}lcccc@{}}
\toprule
\textbf{Critère} & \textbf{BeeWAF v6.0} & \textbf{F5 BIG-IP ASM} & \textbf{ModSecurity CRS} \\
\midrule
Score de détection & \textbf{98.2/100} & 73/100 & 65/100 \\
Grade & \textbf{A+} & B & C+ \\
Faux Positifs & \textbf{0\%} & $\sim$5\% & $\sim$8\% \\
Règles & \textbf{10\,041} & $\sim$2\,500 & $\sim$900 \\
ML intégré & \textbf{Oui (3 modèles)} & Limité & Non \\
Prix & \textbf{Open Source} & $\sim$\$15\,000/an & Gratuit \\
\bottomrule
\end{tabular}
\end{table}

\section{Public Cible}

\begin{itemize}[leftmargin=2em]
  \item Entreprises nécessitant une protection WAF avancée
  \item Équipes DevSecOps intégrant la sécurité dans le CI/CD
  \item Organisations soumises à des réglementations (PCI DSS, GDPR, HIPAA)
  \item Laboratoires de recherche en cybersécurité
\end{itemize}


% ============================================================================
% CHAPITRE 2 : OBJECTIFS DU PROJET
% ============================================================================
\chapter{Objectifs du Projet}

\section{Objectifs Fonctionnels}

\begin{table}[H]
\centering
\caption{Objectifs fonctionnels}
\label{tab:objectifs-fonctionnels}
\small
\begin{tabularx}{\textwidth}{@{}c L{8cm} c@{}}
\toprule
\textbf{ID} & \textbf{Objectif} & \textbf{Statut} \\
\midrule
OF-01 & Détecter $\geq$95\% des attaques web connues (OWASP Top 10) & \textcolor{beegreen}{\checkmark} 98.2\% \\
OF-02 & Maintenir un taux de faux positifs $\leq$2\% & \textcolor{beegreen}{\checkmark} 0\% \\
OF-03 & Supporter les protocoles HTTP/1.1 et HTTPS (TLS 1.2/1.3) & \textcolor{beegreen}{\checkmark} \\
OF-04 & Fournir une API REST d'administration sécurisée & \textcolor{beegreen}{\checkmark} 14 endpoints \\
OF-05 & Intégrer un moteur ML adaptatif auto-apprenant & \textcolor{beegreen}{\checkmark} 3 modèles \\
OF-06 & Générer des logs structurés exploitables (ELK) & \textcolor{beegreen}{\checkmark} JSON \\
OF-07 & Être déployable en conteneurs (Docker/K8s) & \textcolor{beegreen}{\checkmark} 6 services \\
OF-08 & Couvrir $\geq$5 frameworks de conformité & \textcolor{beegreen}{\checkmark} 7 frameworks \\
OF-09 & Protéger contre les attaques zero-day & \textcolor{beegreen}{\checkmark} 9 facteurs \\
OF-10 & Supporter le mode clustering multi-n\oe{}uds & \textcolor{beegreen}{\checkmark} \\
\bottomrule
\end{tabularx}
\end{table}

\section{Objectifs Non-Fonctionnels}

\begin{table}[H]
\centering
\caption{Objectifs non-fonctionnels}
\label{tab:objectifs-non-fonctionnels}
\begin{tabular}{@{}clccc@{}}
\toprule
\textbf{ID} & \textbf{Objectif} & \textbf{Cible} & \textbf{Réalisé} & \textbf{Statut} \\
\midrule
ONF-01 & Latence de traitement & $\leq$50ms P99 & 18ms P99 & \textcolor{beegreen}{\checkmark} \\
ONF-02 & Temps de détection d'attaque & $\leq$20ms & 11ms avg & \textcolor{beegreen}{\checkmark} \\
ONF-03 & Disponibilité & 99.9\% & 99.9\% & \textcolor{beegreen}{\checkmark} \\
ONF-04 & Consommation mémoire & $\leq$512 Mo & $<$512 Mo & \textcolor{beegreen}{\checkmark} \\
ONF-05 & Démarrage à froid & $\leq$15s & $\sim$12s & \textcolor{beegreen}{\checkmark} \\
\bottomrule
\end{tabular}
\end{table}


% ============================================================================
% CHAPITRE 3 : ARCHITECTURE GÉNÉRALE
% ============================================================================
\chapter{Architecture Générale}

\section{Diagramme d'Architecture}

\begin{figure}[H]
\centering
\begin{tikzpicture}[
  box/.style={draw, rounded corners=5pt, minimum width=3cm, minimum height=1.2cm, font=\small, align=center},
  arrow/.style={-{Stealth[length=6pt]}, thick},
  node distance=1.5cm and 2cm
]
  % Client
  \node[box, fill=gray!15] (client) {Client\\HTTPS};

  % Nginx
  \node[box, fill=beeblue!20, right=2cm of client] (nginx) {Nginx\\:80/:443\\TLS Termination};

  % BeeWAF
  \node[box, fill=beeyellow!25, right=2cm of nginx] (beewaf) {BeeWAF Core\\FastAPI :8000};

  % Backend
  \node[box, fill=beegreen!20, right=2cm of beewaf] (backend) {Backend\\Application};

  % Modules inside BeeWAF
  \node[box, fill=beeyellow!10, below=0.8cm of beewaf, minimum width=2.5cm] (modules) {27 Modules\\Sécurité};
  \node[box, fill=beeyellow!10, below=0.5cm of modules, minimum width=2.5cm] (regex) {10\,041\\Règles Regex};
  \node[box, fill=beeyellow!10, below=0.5cm of regex, minimum width=2.5cm] (ml) {ML Engine\\3 Modèles};

  % ELK
  \node[box, fill=Orchid!20, below=0.8cm of backend] (es) {Elasticsearch\\:9200};
  \node[box, fill=Orchid!20, below=0.5cm of es] (kibana) {Kibana\\:5601};
  \node[box, fill=Orchid!20, left=1.5cm of es] (logstash) {Logstash\\:5044};
  \node[box, fill=Orchid!20, left=1.5cm of logstash] (filebeat) {Filebeat};

  % Arrows
  \draw[arrow] (client) -- (nginx);
  \draw[arrow] (nginx) -- (beewaf);
  \draw[arrow] (beewaf) -- (backend);
  \draw[arrow, dashed, gray] (beewaf) -- (modules);
  \draw[arrow, dashed, gray] (modules) -- (regex);
  \draw[arrow, dashed, gray] (regex) -- (ml);
  \draw[arrow, dashed, beeblue] (beewaf) -- node[right, font=\tiny]{logs JSON} (filebeat);
  \draw[arrow, dashed, beeblue] (filebeat) -- (logstash);
  \draw[arrow, dashed, beeblue] (logstash) -- (es);
  \draw[arrow, dashed, beeblue] (es) -- (kibana);

\end{tikzpicture}
\caption{Architecture générale du cluster BeeWAF}
\label{fig:architecture}
\end{figure}

\section{Flux de Traitement --- 36 Étapes Séquentielles}

Le middleware WAF exécute les étapes suivantes dans l'ordre :

\begin{enumerate}[leftmargin=2em]
  \item IP Blacklist Check
  \item Path Normalization (URL decode, \texttt{//}, \texttt{..})
  \item Host Header Validation
  \item Sensitive Path Blocking
  \item X-Forwarded-For Spoof Detection
  \item Negative ID Detection
  \item Transfer-Encoding Smuggling
  \item Range Header Validation
  \item Business Logic Body Checks
  \item Protocol Validator
  \item Bot Detector / Bot Manager Advanced
  \item DDoS Protection
  \item Rate Limiting
  \item Threat Intelligence
  \item Threat Feed
  \item Session Protection
  \item API Security (JSON/XML/GraphQL)
  \item Evasion Detector (18 couches de déobfuscation)
  \item Correlation Engine
  \item Adaptive Learning
  \item Cookie Security
  \item Virtual Patching (37 CVE)
  \item Zero-Day Detector
  \item WebSocket Inspector
  \item Payload Analyzer
  \item API Discovery
  \item Header Validation (Referer, Cookie, X-*)
  \item \textbf{Regex Rules Check (10\,041 patterns)}
  \item \textbf{ML Engine Check (ensemble 3 modèles)}
  \item DLP Scanning (Response)
  \item Response Cloaking
  \item Compliance Engine Logging
  \item Prometheus Metrics Update
  \item ELK Structured Logging
  \item Geo-IP Enrichment
  \item Cluster Sync
\end{enumerate}

\noindent Résultat : \textbf{403 Blocked} ou passage au backend.


% ============================================================================
% CHAPITRE 4 : STACK TECHNOLOGIQUE
% ============================================================================
\chapter{Stack Technologique}

\section{Langages \& Frameworks}

\begin{table}[H]
\centering
\caption{Stack technologique principale}
\label{tab:stack}
\begin{tabular}{@{}llc@{}}
\toprule
\textbf{Composant} & \textbf{Technologie} & \textbf{Version} \\
\midrule
Langage principal & Python & 3.11 \\
Framework Web & FastAPI & $\geq$ 0.100.0 \\
Serveur ASGI & Uvicorn & $\geq$ 0.22.0 \\
Reverse Proxy & Nginx & 1.29.x (Alpine) \\
Conteneurisation & Docker & 24+ \\
Orchestration & Kubernetes & 1.28+ \\
CI/CD & Jenkins & 2.x \\
Logging & ELK Stack & 8.11.0 \\
Monitoring & Prometheus & Compatible \\
\bottomrule
\end{tabular}
\end{table}

\section{Bibliothèques Python}

\begin{table}[H]
\centering
\caption{Dépendances Python}
\label{tab:python-deps}
\small
\begin{tabularx}{\textwidth}{@{}llcX@{}}
\toprule
\textbf{Catégorie} & \textbf{Package} & \textbf{Version} & \textbf{Rôle} \\
\midrule
\multirow{4}{*}{\textbf{Core}} & \texttt{fastapi} & $\geq$ 0.100.0 & Framework API REST \\
 & \texttt{uvicorn[standard]} & $\geq$ 0.22.0 & Serveur ASGI haute performance \\
 & \texttt{python-multipart} & $\geq$ 0.0.6 & Parsing multipart/form-data \\
 & \texttt{aiofiles} & $\geq$ 23.0.0 & I/O fichier asynchrone \\
\midrule
\multirow{2}{*}{\textbf{HTTP}} & \texttt{requests} & $\geq$ 2.31.0 & Client HTTP synchrone \\
 & \texttt{httpx} & $\geq$ 0.24.0 & Client HTTP asynchrone \\
\midrule
\multirow{4}{*}{\textbf{ML}} & \texttt{numpy} & $\geq$ 1.24.0 & Calcul numérique \\
 & \texttt{scipy} & $\geq$ 1.11.0 & Fonctions scientifiques \\
 & \texttt{scikit-learn} & $\geq$ 1.3.0 & Algorithmes ML \\
 & \texttt{joblib} & $\geq$ 1.3.0 & Sérialisation modèles \\
\midrule
\textbf{Monitoring} & \texttt{prometheus-client} & $\geq$ 0.17.0 & Métriques Prometheus \\
\textbf{Logging} & \texttt{python-json-logger} & $\geq$ 2.0.7 & Logs JSON structurés \\
\textbf{Optionnel} & \texttt{clamd} & $\geq$ 1.0.2 & Intégration ClamAV \\
\bottomrule
\end{tabularx}
\end{table}


% ============================================================================
% CHAPITRE 5 : MODULES DE SÉCURITÉ
% ============================================================================
\chapter{Modules de Sécurité (27 Modules)}

\section{Tableau Récapitulatif}

\begin{longtable}{@{}clllc@{}}
\caption{Les 27 modules de sécurité de BeeWAF}
\label{tab:modules}\\
\toprule
\textbf{\#} & \textbf{Module} & \textbf{Fichier} & \textbf{Description} & \textbf{Catégorie} \\
\midrule
\endfirsthead
\toprule
\textbf{\#} & \textbf{Module} & \textbf{Fichier} & \textbf{Description} & \textbf{Catégorie} \\
\midrule
\endhead
\bottomrule
\endfoot
1 & Rules Engine & rules.py + 15 & 10\,041 patterns regex & Détection \\
2 & Anomaly Detector & anomaly.py & IsolationForest (legacy) & ML \\
3 & ML Engine & ml\_engine.py & Ensemble 3 modèles & ML \\
4 & Rate Limiter & ratelimit.py & Limitation débit + IP & Protection \\
5 & Bot Detector & bot\_detector.py & Détection User-Agent & Détection \\
6 & Bot Manager Adv. & bot\_manager*.py & JS Challenge, TLS FP & Détection \\
7 & DLP & dlp.py & Prévention fuite données & Protection \\
8 & Geo Block & geo\_block.py & Blocage géographique & Contrôle \\
9 & Protocol Validator & protocol\_*.py & Validation HTTP stricte & Validation \\
10 & API Security & api\_security.py & JSON/XML/GraphQL & Protection \\
11 & Threat Intel & threat\_intel.py & Intelligence de menaces & Renseign. \\
12 & Threat Feed & threat\_feed.py & Flux menaces externes & Renseign. \\
13 & Session Protection & session\_*.py & Anti-hijacking, JWT & Session \\
14 & Evasion Detector & evasion\_*.py & 18 couches déobfuscation & Détection \\
15 & Correlation Engine & correlation\_*.py & Multi-événements & Analyse \\
16 & Adaptive Learning & adaptive\_*.py & Modèle positif & ML \\
17 & Response Cloaking & response\_*.py & Masquage headers & Protection \\
18 & Cookie Security & cookie\_*.py & HMAC, altération & Session \\
19 & Virtual Patching & virtual\_*.py & 37 patches CVE & Protection \\
20 & Zero-Day Detector & zero\_day\_*.py & Anomalies 9 facteurs & ML \\
21 & WebSocket Insp. & websocket\_*.py & Inspection WS & Détection \\
22 & Payload Analyzer & payload\_*.py & Analyse payload & Détection \\
23 & Compliance Engine & compliance\_*.py & 7 frameworks & Conformité \\
24 & DDoS Protection & ddos\_*.py & Anti-DDoS (RPS) & Protection \\
25 & API Discovery & api\_discovery.py & Shadow API & Découverte \\
26 & Cluster Manager & cluster\_*.py & Multi-n\oe{}uds & Infra. \\
27 & Performance Engine & performance\_*.py & Cache, bloom filter & Performance \\
\end{longtable}

\section{Détail des Modules Clés}

\subsection{Bot Detector / Bot Manager Advanced}

\begin{itemize}[leftmargin=2em]
  \item Détection de 100+ User-Agents de scanners (SQLMap, Nikto, Nmap, Masscan, Acunetix, Burp Suite, etc.)
  \item Détection User-Agent vide ou suspect
  \item Challenge JavaScript (Bot Manager Advanced)
  \item Fingerprint TLS / JA3
  \item Détection credential stuffing (seuil : 5 tentatives/60s)
  \item Classification : bon bot, mauvais bot, bot suspect
\end{itemize}

\subsection{DLP (Data Loss Prevention)}

\textbf{Données protégées :}
\begin{itemize}[leftmargin=2em]
  \item Numéros de carte bancaire (Visa, Mastercard, Amex)
  \item Numéros de sécurité sociale (SSN)
  \item Adresses email
  \item Numéros de téléphone
  \item Données médicales (HIPAA)
\end{itemize}

\noindent\textbf{Mode :} Scan bidirectionnel (requête + réponse).

\subsection{Evasion Detector --- 18 Couches de Déobfuscation}

\begin{table}[H]
\centering
\caption{Les 18 couches de déobfuscation}
\label{tab:evasion}
\small
\begin{tabular}{@{}cl@{}}
\toprule
\textbf{Couche} & \textbf{Technique} \\
\midrule
1 & URL Decoding (simple) \\
2 & Double URL Decoding \\
3 & Triple URL Decoding \\
4 & HTML Entity Decoding \\
5 & Unicode Normalization (NFD $\to$ NFC) \\
6 & UTF-8 Overlong Decoding \\
7 & Hex Escape Decoding (\texttt{\textbackslash x41}) \\
8 & Octal Escape Decoding (\texttt{\textbackslash 101}) \\
9 & Base64 Decoding \\
10 & Mixed Case Normalization \\
11 & Null Byte Removal \\
12 & Comment Stripping (\texttt{/* */}, \texttt{//}, \texttt{-{}-}) \\
13 & Whitespace Normalization \\
14 & Backslash Normalization \\
15 & Tab/Newline Removal \\
16 & Full-Width Character Normalization \\
17 & IIS-specific Decoding (\texttt{\%u00XX}) \\
18 & Path Canonicalization \\
\bottomrule
\end{tabular}
\end{table}

\subsection{Virtual Patching --- 37 CVE Couverts}

\begin{table}[H]
\centering
\caption{Principaux CVE couverts par Virtual Patching}
\label{tab:cve}
\small
\begin{tabular}{@{}llc@{}}
\toprule
\textbf{CVE} & \textbf{Nom} & \textbf{Sévérité} \\
\midrule
CVE-2021-44228 & Log4Shell (Log4j) & Critique \\
CVE-2017-5638 & Apache Struts2 RCE & Critique \\
CVE-2022-22965 & Spring4Shell & Critique \\
CVE-2021-26855 & ProxyLogon (Exchange) & Critique \\
CVE-2021-34473 & ProxyShell & Critique \\
CVE-2023-34362 & MOVEit Transfer SQLi & Critique \\
CVE-2023-44228 & Apache ActiveMQ RCE & Critique \\
CVE-2024-3400 & PAN-OS GlobalProtect & Critique \\
CVE-2023-46747 & F5 BIG-IP Auth Bypass & Critique \\
CVE-2021-41773 & Apache Path Traversal & Haute \\
\multicolumn{3}{c}{\emph{+ 27 autres CVE de sévérité Haute/Critique}} \\
\bottomrule
\end{tabular}
\end{table}

\subsection{Correlation Engine}

\textbf{Chaînes d'attaques détectées :}
\begin{itemize}[leftmargin=2em]
  \item Reconnaissance $\to$ Exploitation $\to$ Exfiltration
  \item Scanner probe $\to$ Info disclosure $\to$ Data extraction
  \item Brute force $\to$ Auth bypass $\to$ Privilege escalation
  \item XSS $\to$ Session hijacking $\to$ Account takeover
  \item SQLi $\to$ Data extraction $\to$ Command execution
  \item GraphQL introspection $\to$ Scanner probe
  \item SSRF $\to$ Cloud metadata $\to$ Credential theft
\end{itemize}

\subsection{DDoS Protection}

\begin{table}[H]
\centering
\caption{Seuils de protection DDoS}
\label{tab:ddos}
\begin{tabular}{@{}lc@{}}
\toprule
\textbf{Paramètre} & \textbf{Seuil} \\
\midrule
Avertissement RPS & 500 req/s \\
Throttling RPS & 800 req/s \\
Blocage RPS & 1\,000 req/s \\
Max connexions/IP & 100\,000 \\
Fenêtre d'analyse & 60 secondes \\
\bottomrule
\end{tabular}
\end{table}

\subsection{Cookie Security}

\begin{itemize}[leftmargin=2em]
  \item Inspection des valeurs de cookies pour SQLi/XSS
  \item Détection d'altération de cookies de session
  \item Vérification HMAC pour intégrité
  \item Détection de fixation de session
\end{itemize}


% ============================================================================
% CHAPITRE 6 : MOTEUR DE RÈGLES REGEX
% ============================================================================
\chapter{Moteur de Règles Regex (10\,041 Règles)}

\section{Architecture des Fichiers de Règles}

\begin{table}[H]
\centering
\caption{Distribution des règles par fichier}
\label{tab:rules-files}
\small
\begin{tabular}{@{}llr@{}}
\toprule
\textbf{Fichier} & \textbf{Catégories} & \textbf{Nombre} \\
\midrule
rules.py (base) & SQLi, XSS, CMDi, Path Traversal, SSRF & $\sim$287 \\
rules\_extended.py & 26 catégories avancées & 586 \\
rules\_advanced.py & 13 catégories (cloud, k8s, OAuth) & 425 \\
rules\_v5.py & 31 nouvelles catégories & 1\,207 \\
rules\_mega\_1.py & Deep SQLi, Deep XSS & 1\,120 \\
rules\_mega\_2.py & CMS, Framework attacks & 542 \\
rules\_mega\_3.py & Encoding evasion deep & 412 \\
rules\_mega\_4.py & Emerging threats & 292 \\
rules\_mega\_5.py & Protocol attacks & 313 \\
rules\_mega\_6.py & Infrastructure/cloud deep & 214 \\
rules\_mega\_7.py & Scanner fingerprints, SSTI deep & 1\,091 \\
rules\_mega\_8.py & API endpoint, miscellaneous & 1\,161 \\
rules\_mega\_9.py & Advanced patterns & 887 \\
rules\_mega\_10.py & Extended coverage & 776 \\
rules\_mega\_11.py & Specialized attacks & 576 \\
rules\_mega\_12.py & Final coverage & 152 \\
\midrule
\textbf{TOTAL} & & \textbf{10\,041} \\
\bottomrule
\end{tabular}
\end{table}

\section{Catégories d'Attaques Couvertes (50+)}

\begin{longtable}{@{}L{3.5cm}L{11cm}@{}}
\caption{Catégories d'attaques et sous-types}
\label{tab:attack-categories}\\
\toprule
\textbf{Catégorie} & \textbf{Sous-types} \\
\midrule
\endfirsthead
\toprule
\textbf{Catégorie} & \textbf{Sous-types} \\
\midrule
\endhead
\bottomrule
\endfoot
SQL Injection & UNION-based, Blind (Boolean/Time), Error-based, Stacked queries, Hex encoding, Unicode, Information Schema, Out-of-band \\
XSS & Reflected, Stored, DOM-based, SVG, Data URI, Event handlers, JSFuck, Polyglot \\
Command Injection & Semicolon, Pipe, Backtick, Dollar substitution, Wget/Curl, Python/Perl/Ruby \\
Path Traversal & Basic (\texttt{../}), URL-encoded, Double-encoded, Windows, Unicode, Overlong UTF-8 \\
SSRF & AWS IMDSv1/v2, GCP Metadata, Azure IMDS, K8s API, Docker socket, DNS rebind, IPv6 \\
XXE & Entity injection, DOCTYPE, Parameter entity, Billion laughs, Out-of-band \\
SSTI & Jinja2, Twig, Freemarker, Thymeleaf, Velocity, Pebble, Smarty \\
Deserialization & Java (ObjectInputStream), PHP (unserialize), Python (pickle), .NET, YAML, Ruby \\
LDAP Injection & OR injection, Filter manipulation, Wildcard exploitation \\
NoSQL Injection & MongoDB \$ne/\$gt/\$regex/\$where, Aggregation pipeline \\
XPath Injection & Boolean-based, Error-based \\
GraphQL & Introspection, Depth attacks, Batch queries, Aliases \\
JWT Attacks & alg:none, Key confusion, Claim manipulation \\
CRLF Injection & Header injection, HTTP response splitting \\
Open Redirect & URL parameter manipulation \\
CSV/Formula Inj. & DDE injection, =CMD() \\
Prototype Pollution & \texttt{\_\_proto\_\_}, \texttt{constructor.prototype} \\
File Upload & PHP webshell, JSP shell, Double extension, Polyglot \\
CMS Attacks & WordPress, Joomla, Drupal, Magento \\
Cloud/K8s & AWS, GCP, Azure, Kubernetes secrets/API \\
CI/CD & Jenkins, GitLab CI, GitHub Actions \\
Encoding Evasion & Double encoding, Unicode tricks, Hex, Overlong UTF-8 \\
WAF Bypass & Obfuscation, Alternative encodings, Comment insertion \\
Scanner Fingerprints & 200+ outils de scan reconnus \\
\end{longtable}

\section{Compilation \& Optimisation}

\begin{lstlisting}[language=Python, caption={Compilation des règles au démarrage}]
# Toutes les règles sont pré-compilées au démarrage
COMPILED_RULES: List[Tuple[re.Pattern, str]] = []

# Chaque pattern est compilé avec re.IGNORECASE
for regex_str, category in all_patterns:
    COMPILED_RULES.append(
        (re.compile(regex_str, re.IGNORECASE), category.lower())
    )
\end{lstlisting}

\textbf{Optimisations :}
\begin{itemize}[leftmargin=2em]
  \item Cache LRU pour les patterns fréquemment matchés
  \item Bloom filter pour pré-screening des requêtes sûres
  \item Déduplication des requêtes identiques
  \item Short-circuit : arrêt au premier match
\end{itemize}

\section{API Publique}

\begin{lstlisting}[language=Python, caption={API du moteur de règles}]
def check_regex_rules(path: str, body: str,
                      headers: Dict) -> Tuple[bool, str]:
    """
    Vérifie une requête contre les 10 041 règles regex.
    Returns: (is_blocked, rule_category)
    """

def list_rules() -> List[Tuple[str, str]]:
    """Retourne toutes les règles: [(pattern, category), ...]"""
\end{lstlisting}


% ============================================================================
% CHAPITRE 7 : MOTEUR ML — INTELLIGENCE ARTIFICIELLE
% ============================================================================
\chapter{Moteur ML --- Intelligence Artificielle}

\section{Architecture de l'Ensemble}

\begin{figure}[H]
\centering
\begin{tikzpicture}[
  box/.style={draw, rounded corners=3pt, minimum width=2.8cm, minimum height=1cm, font=\small, align=center},
  arrow/.style={-{Stealth[length=5pt]}, thick},
  node distance=1cm and 1.5cm
]
  \node[box, fill=gray!10] (input) {Requête HTTP};
  \node[box, fill=beeblue!15, below=0.8cm of input, minimum width=10cm] (features) {Feature Extractor\\(35 features)};

  \node[box, fill=beered!15, below left=1.2cm and 1.5cm of features] (if) {IsolationForest\\Poids : 0.10};
  \node[box, fill=beegreen!15, below=1.2cm of features] (rf) {RandomForest\\Poids : 0.45};
  \node[box, fill=beeblue!15, below right=1.2cm and 1.5cm of features] (gb) {GradientBoosting\\Poids : 0.45};

  \node[box, fill=beeyellow!25, below=1.2cm of rf, minimum width=8cm] (ensemble) {Score Pondéré = $\sum$(weight $\times$ prediction)\\Si score $> 0.6$ $\to$ \textbf{ATTAQUE DÉTECTÉE}};

  \draw[arrow] (input) -- (features);
  \draw[arrow] (features) -- (if);
  \draw[arrow] (features) -- (rf);
  \draw[arrow] (features) -- (gb);
  \draw[arrow] (if) -- (ensemble);
  \draw[arrow] (rf) -- (ensemble);
  \draw[arrow] (gb) -- (ensemble);
\end{tikzpicture}
\caption{Architecture de l'ensemble ML à 3 modèles}
\label{fig:ml-ensemble}
\end{figure}

\section{Modèles ML}

\begin{table}[H]
\centering
\caption{Performance des modèles ML}
\label{tab:ml-models}
\begin{tabular}{@{}llcccc@{}}
\toprule
\textbf{Modèle} & \textbf{Algorithme} & \textbf{Poids} & \textbf{Accuracy} & \textbf{F1} & \textbf{ROC-AUC} \\
\midrule
Model 1 & IsolationForest & 0.10 & 77.3\% & 0.724 & --- \\
Model 2 & RandomForest & 0.45 & 94.2\% & 0.932 & 0.992 \\
Model 3 & GradientBoosting & 0.45 & 95.3\% & 0.943 & 0.993 \\
\midrule
\textbf{Ensemble} & \textbf{Weighted Average} & \textbf{1.00} & \textbf{96.8\%} & \textbf{0.954} & --- \\
\bottomrule
\end{tabular}
\end{table}

\section{Extraction de Features (35 Features)}

\subsection{Groupe 1 --- Longueur (6 features)}

\begin{table}[H]
\centering
\small
\begin{tabular}{@{}ll@{}}
\toprule
\textbf{Feature} & \textbf{Description} \\
\midrule
\texttt{url\_length} & Longueur totale de l'URL \\
\texttt{path\_length} & Longueur du chemin \\
\texttt{query\_length} & Longueur de la query string \\
\texttt{body\_length} & Longueur du body \\
\texttt{header\_count} & Nombre de headers HTTP \\
\texttt{cookie\_length} & Longueur totale des cookies \\
\bottomrule
\end{tabular}
\end{table}

\subsection{Groupe 2 --- Distribution de Caractères (8 features)}

\begin{table}[H]
\centering
\small
\begin{tabular}{@{}ll@{}}
\toprule
\textbf{Feature} & \textbf{Description} \\
\midrule
\texttt{special\_char\_count} & Nombre de caractères spéciaux \\
\texttt{special\_char\_ratio} & Ratio caractères spéciaux / total \\
\texttt{dangerous\_char\_score} & Score pondéré des chars dangereux (\texttt{'}, \texttt{"}, \texttt{<}, \texttt{>}, \texttt{;}) \\
\texttt{uppercase\_ratio} & Ratio majuscules \\
\texttt{digit\_ratio} & Ratio chiffres \\
\texttt{non\_ascii\_count} & Nombre de caractères non-ASCII \\
\texttt{max\_char\_repeat} & Plus longue répétition d'un caractère \\
\texttt{entropy} & Entropie de Shannon de la requête \\
\bottomrule
\end{tabular}
\end{table}

\subsection{Groupe 3 --- Mots-clés (5 features)}

\begin{table}[H]
\centering
\small
\begin{tabular}{@{}ll@{}}
\toprule
\textbf{Feature} & \textbf{Description} \\
\midrule
\texttt{sql\_keyword\_count} & Nombre de mots-clés SQL (SELECT, UNION, etc.) \\
\texttt{xss\_keyword\_count} & Nombre de mots-clés XSS (script, alert, etc.) \\
\texttt{cmd\_keyword\_count} & Nombre de mots-clés commande (cat, wget, etc.) \\
\texttt{path\_traversal\_count} & Nombre de séquences \texttt{../} \\
\texttt{ssrf\_keyword\_count} & Nombre de mots-clés SSRF (169.254, metadata, etc.) \\
\bottomrule
\end{tabular}
\end{table}

\subsection{Groupe 4 --- Encodage (4 features)}

\begin{table}[H]
\centering
\small
\begin{tabular}{@{}ll@{}}
\toprule
\textbf{Feature} & \textbf{Description} \\
\midrule
\texttt{url\_encoding\_count} & Nombre de séquences \%XX \\
\texttt{double\_encoding\_count} & Nombre de double-encodages \%25XX \\
\texttt{hex\_encoding\_count} & Nombre de séquences \texttt{\textbackslash xXX} ou \texttt{0xXX} \\
\texttt{unicode\_encoding\_count} & Nombre de séquences \texttt{\textbackslash uXXXX} \\
\bottomrule
\end{tabular}
\end{table}

\subsection{Groupe 5 --- Structure (7 features)}

\begin{table}[H]
\centering
\small
\begin{tabular}{@{}ll@{}}
\toprule
\textbf{Feature} & \textbf{Description} \\
\midrule
\texttt{param\_count} & Nombre de paramètres query/body \\
\texttt{nested\_bracket\_depth} & Profondeur d'imbrication des parenthèses/crochets \\
\texttt{comment\_patterns} & Nombre de patterns de commentaires \\
\texttt{null\_byte\_count} & Nombre de null bytes (\%00) \\
\texttt{whitespace\_anomaly} & Score d'anomalie des espaces \\
\texttt{method\_encoded} & Méthode HTTP encodée (booléen) \\
\texttt{suspicious\_extension} & Extension de fichier suspecte \\
\bottomrule
\end{tabular}
\end{table}

\subsection{Groupe 6 --- Contexte (5 features)}

\begin{table}[H]
\centering
\small
\begin{tabular}{@{}ll@{}}
\toprule
\textbf{Feature} & \textbf{Description} \\
\midrule
\texttt{has\_valid\_tld} & URL contient un TLD valide \\
\texttt{path\_depth} & Profondeur du chemin (nombre de /) \\
\texttt{query\_key\_anomaly} & Anomalie dans les noms de paramètres \\
\texttt{body\_is\_json} & Body est du JSON valide \\
\texttt{mixed\_case\_keywords} & Présence de mots-clés en casse mixte \\
\bottomrule
\end{tabular}
\end{table}

\section{Pré-filtrage Intelligent}

Avant l'inférence ML (coûteuse), un pré-filtre identifie les requêtes évidemment sûres :

\begin{enumerate}[leftmargin=2em]
  \item \textbf{JSON valide} sans mots-clés d'attaque $\to$ SAFE
  \item \textbf{Patterns dangereux} (\texttt{<}, \texttt{>}, \texttt{;}, \texttt{|}, etc.) $\to$ ANALYZE
  \item \textbf{Mots dangereux} (script, alert) avec \emph{word boundaries} $\to$ ANALYZE
  \item \textbf{Contexte SQL} (SELECT + FROM, UNION + SELECT) $\to$ ANALYZE
  \item \textbf{Apostrophes en contexte SQL} (\texttt{' OR}, \texttt{' AND}, \texttt{'='='}) $\to$ ANALYZE
  \item \textbf{Extensions statiques} (.html, .css, .js, .png, .jpg) $\to$ SAFE
  \item \textbf{Chemins simples} ($<$100 chars, $\leq$5 niveaux) $\to$ SAFE
  \item \textbf{Query params simples} (alphanumérique + \texttt{=\&\_-+.\%,:}) $\to$ SAFE
\end{enumerate}

\section{Détermination du Type d'Attaque}

En cas de détection, le système classifie automatiquement :

\begin{table}[H]
\centering
\begin{tabular}{@{}ll@{}}
\toprule
\textbf{Label} & \textbf{Type d'attaque} \\
\midrule
\texttt{sqli} & Injection SQL \\
\texttt{xss} & Cross-Site Scripting \\
\texttt{cmdi} & Injection de commande \\
\texttt{path\_traversal} & Traversée de répertoire \\
\texttt{ssrf} & Server-Side Request Forgery \\
\texttt{injection} & Injection générique \\
\texttt{suspicious} & Activité suspecte \\
\texttt{anomaly} & Anomalie non classifiée \\
\bottomrule
\end{tabular}
\end{table}

\section{Données d'Entraînement}

\begin{table}[H]
\centering
\caption{Jeux de données d'entraînement}
\label{tab:datasets}
\begin{tabular}{@{}llcl@{}}
\toprule
\textbf{Dataset} & \textbf{Fichier} & \textbf{Taille} & \textbf{Usage} \\
\midrule
CSIC 2010 & csic\_database.csv & $\sim$61\,065 éch. & Entraînement principal ML \\
Train Demo & train\_demo.csv & Petit & Demo/test anomaly legacy \\
Train Kaggle & train\_kaggle.csv & Variable & Enrichissement \\
Train Synthetic & train\_synthetic.csv & Variable & Données synthétiques \\
\bottomrule
\end{tabular}
\end{table}

\noindent\textbf{Split :} 80\% train / 20\% test \\
\textbf{Attack ratio :} $\sim$41\% des échantillons sont des attaques

\section{Persistance des Modèles}

\begin{lstlisting}[language=bash, caption={Structure des fichiers modèles}]
models/
  anomaly_model.pkl    # Legacy IsolationForest (via pickle)
  ml_model.pkl         # Ensemble 3 modeles (via joblib/pickle)
\end{lstlisting}


% ============================================================================
% CHAPITRE 8 : PIPELINE DE TRAITEMENT DES REQUÊTES
% ============================================================================
\chapter{Pipeline de Traitement des Requêtes}

\section{Middleware WAF --- Ordre d'Exécution}

\begin{lstlisting}[language=Python, caption={Point d'entrée du middleware WAF}]
@app.middleware("http")
async def waf_middleware(request: Request, call_next):
\end{lstlisting}

Le traitement s'organise en 7 phases :

\begin{description}[leftmargin=2em]
  \item[Phase 1 --- Pré-validation :] Extraction IP, path, method, headers, body ; incrémentation compteur Prometheus ; vérification IP blacklist ; normalisation du path.
  \item[Phase 2 --- Validation d'en-têtes :] Host header, chemins sensibles, X-Forwarded-For, ID négatifs, Transfer-Encoding smuggling, Range header.
  \item[Phase 3 --- Logique métier :] Password reset IDOR, quantity abuse.
  \item[Phase 4 --- Modules enterprise :] 16 modules spécialisés (Protocol Validator $\to$ API Discovery).
  \item[Phase 5 --- Header scanning :] Scan des headers sélectifs (Referer, Cookie, X-Original-URL).
  \item[Phase 6 --- Détection principale :] Regex Rules Engine (10\,041 patterns) puis ML Engine (ensemble 3 modèles).
  \item[Phase 7 --- Post-traitement :] Passage au backend, DLP Response Scanning, Response Cloaking, Logging ELK, Métriques Prometheus, Compliance Engine audit.
\end{description}

\section{Format de Réponse de Blocage}

\begin{lstlisting}[language=Python, caption={Format JSON de blocage}]
{
    "blocked": true,
    "reason": "regex-sqli"
}
\end{lstlisting}

\textbf{Catégories de blocage :}

\begin{table}[H]
\centering
\small
\begin{tabular}{@{}ll@{}}
\toprule
\textbf{Préfixe} & \textbf{Description} \\
\midrule
\texttt{regex-\{category\}} & Détecté par une règle regex \\
\texttt{ml-\{attack\_type\}} & Détecté par le moteur ML \\
\texttt{bot-detected} & Bot malveillant \\
\texttt{rate-limited} & Dépassement de seuil \\
\texttt{ddos-detected} & Attaque DDoS \\
\texttt{ip-blocked} & IP en liste noire \\
\texttt{virtual-patch-\{cve\}} & Patch virtuel CVE \\
\texttt{business-logic-\{type\}} & Règle logique métier \\
\texttt{sensitive-path} & Chemin sensible bloqué \\
\texttt{xff-spoof} & Spoofing X-Forwarded-For \\
\texttt{negative-id} & ID négatif dans URL \\
\texttt{te-smuggling} & Smuggling Transfer-Encoding \\
\bottomrule
\end{tabular}
\end{table}


% ============================================================================
% CHAPITRE 9 : API REST & ENDPOINTS
% ============================================================================
\chapter{API REST \& Endpoints}

\section{Endpoints Publics}

\begin{table}[H]
\centering
\caption{Endpoints publics}
\label{tab:endpoints-public}
\begin{tabularx}{\textwidth}{@{}llX@{}}
\toprule
\textbf{Méthode} & \textbf{Chemin} & \textbf{Description} \\
\midrule
\texttt{GET} & \texttt{/} & Information du service (version, modules, rules\_count, ml\_mode, compliance) \\
\texttt{GET} & \texttt{/health} & Vérification santé (status, ml\_engine\_trained, rules\_count) \\
\texttt{GET} & \texttt{/metrics} & Métriques Prometheus (text/plain) \\
\texttt{POST} & \texttt{/echo} & Echo (test WAF traversal) \\
\bottomrule
\end{tabularx}
\end{table}

\section{Endpoints Admin (API Key requise)}

\begin{table}[H]
\centering
\caption{Endpoints d'administration}
\label{tab:endpoints-admin}
\small
\begin{tabularx}{\textwidth}{@{}llX@{}}
\toprule
\textbf{Méthode} & \textbf{Chemin} & \textbf{Description} \\
\midrule
\texttt{GET} & \texttt{/admin/rules} & Liste toutes les règles compilées \\
\texttt{GET} & \texttt{/admin/ml-stats} & Statistiques ML : modèles, accuracy, weights \\
\texttt{POST} & \texttt{/admin/ml-predict} & Test de prédiction ML sur un payload \\
\texttt{POST} & \texttt{/admin/retrain} & Réentraîner le modèle anomaly legacy \\
\texttt{POST} & \texttt{/admin/retrain-ml} & Réentraîner l'ensemble ML depuis CSIC \\
\texttt{GET} & \texttt{/admin/enterprise-stats} & Stats de tous les 27 modules \\
\texttt{GET} & \texttt{/admin/compliance} & Rapport de conformité 7 frameworks \\
\texttt{GET} & \texttt{/admin/virtual-patches} & Liste des 37 patches virtuels \\
\texttt{GET} & \texttt{/admin/correlation} & Stats corrélation + campagnes actives \\
\texttt{POST} & \texttt{/admin/adaptive-mode} & Changer le mode : learning/detect/enforce \\
\bottomrule
\end{tabularx}
\end{table}

\section{Authentification API}

\begin{lstlisting}[language=bash, caption={En-tête d'authentification}]
Header: X-API-Key: <cle>
Variable: BEEWAF_API_KEY (defaut: changeme-default-key-not-secure)
\end{lstlisting}

\begin{itemize}[leftmargin=2em]
  \item Clé invalide $\to$ \texttt{403 Forbidden}
  \item Clé absente $\to$ \texttt{401 Unauthorized}
\end{itemize}


% ============================================================================
% CHAPITRE 10 : INFRASTRUCTURE DOCKER
% ============================================================================
\chapter{Infrastructure Docker}

\section{Fichiers Docker}

\begin{table}[H]
\centering
\caption{Fichiers Docker disponibles}
\label{tab:dockerfiles}
\begin{tabularx}{\textwidth}{@{}llX@{}}
\toprule
\textbf{Fichier} & \textbf{Image de Base} & \textbf{Usage} \\
\midrule
Dockerfile & python:3.11-slim & Build avec ClamAV \\
\textbf{Dockerfile.full} & \textbf{python:3.11-slim} & \textbf{Build complet (principal)} \\
Dockerfile.runtime & python:3.11-slim & Build léger production \\
Dockerfile.final & python:3.11-slim & Build runtime avec ClamAV \\
\bottomrule
\end{tabularx}
\end{table}

\section{Services Docker Compose (6 conteneurs)}

\begin{table}[H]
\centering
\caption{Services Docker Compose}
\label{tab:docker-services}
\begin{tabular}{@{}lllll@{}}
\toprule
\textbf{Service} & \textbf{Image} & \textbf{Container} & \textbf{Ports} & \textbf{Ressources} \\
\midrule
beewaf & beewaf:sklearn & beewaf\_sklearn & 8000 (expose) & --- \\
nginx & nginx:alpine & beewaf\_nginx & 80, 443 & --- \\
elasticsearch & ES 8.11.0 & beewaf\_elasticsearch & 9200 & 1\,Go heap \\
logstash & LS 8.11.0 & beewaf\_logstash & 5044, 9600 & 256\,Mo heap \\
kibana & Kibana 8.11.0 & beewaf\_kibana & 5601 & --- \\
filebeat & FB 8.11.0 & beewaf\_filebeat & --- & --- \\
\bottomrule
\end{tabular}
\end{table}

\section{Commandes de Déploiement}

\begin{lstlisting}[language=bash, caption={Commandes Docker de déploiement}]
# Build
docker build -f Dockerfile.full -t beewaf:sklearn .

# Demarrage complet (6 services)
docker-compose -f docker-compose-elk.yaml up -d

# Rebuild WAF uniquement
docker-compose -f docker-compose-elk.yaml up -d \
    --force-recreate beewaf

# Logs
docker logs -f beewaf_sklearn
\end{lstlisting}


% ============================================================================
% CHAPITRE 11 : CONFIGURATION NGINX
% ============================================================================
\chapter{Configuration Nginx (Reverse Proxy)}

\section{Paramètres Généraux}

\begin{lstlisting}[language=nginx, caption={Configuration Nginx générale}]
worker_processes  1;
worker_connections  1024;
\end{lstlisting}

\section{Redirection HTTP $\to$ HTTPS}

\begin{lstlisting}[language=nginx, caption={Redirection HTTP vers HTTPS}]
server {
    listen 80;
    server_name _;
    return 301 https://$host$request_uri;
}
\end{lstlisting}

\section{Configuration HTTPS}

\begin{table}[H]
\centering
\caption{Paramètres TLS/SSL}
\label{tab:tls-config}
\begin{tabular}{@{}ll@{}}
\toprule
\textbf{Paramètre} & \textbf{Valeur} \\
\midrule
Port & 443 (SSL) \\
Certificat & \texttt{/etc/nginx/ssl/tls.crt} \\
Clé privée & \texttt{/etc/nginx/ssl/tls.key} \\
Protocoles & TLSv1.2, TLSv1.3 \\
Chiffrement & ECDHE-* (Perfect Forward Secrecy) \\
\bottomrule
\end{tabular}
\end{table}

\section{Headers de Sécurité (Nginx)}

\begin{table}[H]
\centering
\caption{Headers de sécurité ajoutés par Nginx}
\label{tab:nginx-headers}
\begin{tabular}{@{}ll@{}}
\toprule
\textbf{Header} & \textbf{Valeur} \\
\midrule
\texttt{X-Frame-Options} & \texttt{DENY} \\
\texttt{X-Content-Type-Options} & \texttt{nosniff} \\
\texttt{X-XSS-Protection} & \texttt{1; mode=block} \\
\texttt{Strict-Transport-Security} & \texttt{max-age=31536000; includeSubDomains} \\
\bottomrule
\end{tabular}
\end{table}

\section{Proxy Pass}

\begin{lstlisting}[language=nginx, caption={Configuration du reverse proxy Nginx}]
upstream beewaf {
    server beewaf_sklearn:8000;
}

location / {
    proxy_pass http://beewaf;
    proxy_set_header Host $host;
    proxy_set_header X-Real-IP $remote_addr;
    proxy_set_header X-Forwarded-For
        $proxy_add_x_forwarded_for;
    proxy_set_header X-Forwarded-Proto $scheme;
}
\end{lstlisting}


% ============================================================================
% CHAPITRE 12 : STACK ELK
% ============================================================================
\chapter{Stack ELK (Logging \& Monitoring)}

\section{Architecture de Logging}

\begin{figure}[H]
\centering
\begin{tikzpicture}[
  box/.style={draw, rounded corners=3pt, minimum width=3.5cm, minimum height=1cm, font=\small, align=center},
  arrow/.style={-{Stealth[length=5pt]}, thick},
  node distance=1.2cm
]
  \node[box, fill=beeyellow!20] (waf) {BeeWAF\\(JSON logs)};
  \node[box, fill=Orchid!15, below=of waf] (fb) {Filebeat\\(collecte Docker)};
  \node[box, fill=Orchid!15, below=of fb] (ls) {Logstash\\(parsing + enrichissement)};
  \node[box, fill=Orchid!15, below=of ls] (es) {Elasticsearch\\(stockage indexé)};
  \node[box, fill=Orchid!15, below=of es] (kb) {Kibana\\(visualisation)};

  \draw[arrow] (waf) -- (fb);
  \draw[arrow] (fb) -- node[right, font=\scriptsize]{filtre: beewaf\_sklearn} (ls);
  \draw[arrow] (ls) -- node[right, font=\scriptsize]{index: beewaf-logs-YYYY.MM.dd} (es);
  \draw[arrow] (es) -- (kb);
\end{tikzpicture}
\caption{Pipeline de logging ELK}
\label{fig:elk-pipeline}
\end{figure}

\section{Pipeline Logstash}

\begin{table}[H]
\centering
\caption{Étapes du pipeline Logstash}
\label{tab:logstash-pipeline}
\begin{tabularx}{\textwidth}{@{}cX@{}}
\toprule
\textbf{Étape} & \textbf{Action} \\
\midrule
1 & Filtrage : suppression logs non-BeeWAF \\
2 & Suppression : logs d'accès Uvicorn \\
3 & Parsing JSON du champ \texttt{message} \\
4 & Renommage : \texttt{client\_ip}, \texttt{method}, \texttt{path}, \texttt{status\_code}, \texttt{blocked}, \texttt{block\_reason}, \texttt{latency\_ms}, \texttt{body\_preview}, \texttt{user\_agent} \\
5 & Enrichissement : tags d'attaque basés sur \texttt{block\_reason} \\
6 & Nettoyage : suppression métadonnées Filebeat \\
\bottomrule
\end{tabularx}
\end{table}

\section{Index Elasticsearch}

\begin{infobox}[Champs indexés]
\texttt{@timestamp}, \texttt{app\_timestamp}, \texttt{client\_ip}, \texttt{method}, \texttt{path}, \texttt{status\_code}, \texttt{blocked}, \texttt{block\_reason}, \texttt{latency\_ms}, \texttt{body\_preview}, \texttt{user\_agent}, \texttt{service}, \texttt{tags}
\end{infobox}


% ============================================================================
% CHAPITRE 13 : MÉTRIQUES PROMETHEUS
% ============================================================================
\chapter{Métriques Prometheus}

\section{Métriques Exposées}

\begin{table}[H]
\centering
\caption{Métriques Prometheus exposées par BeeWAF}
\label{tab:prometheus}
\begin{tabularx}{\textwidth}{@{}llcX@{}}
\toprule
\textbf{Métrique} & \textbf{Type} & \textbf{Labels} & \textbf{Description} \\
\midrule
\texttt{beewaf\_requests\_total} & Counter & --- & Total de requêtes HTTP \\
\texttt{beewaf\_blocked\_total} & Counter & \texttt{reason} & Total bloquées par catégorie \\
\texttt{beewaf\_request\_latency\_seconds} & Histogram & --- & Distribution de la latence \\
\texttt{beewaf\_active\_requests} & Gauge & --- & Requêtes en cours \\
\texttt{beewaf\_rules\_count} & Gauge & --- & Nombre de règles chargées \\
\texttt{beewaf\_model\_loaded} & Gauge & --- & Statut modèle ML (0/1) \\
\bottomrule
\end{tabularx}
\end{table}

\section{Endpoint}

\begin{lstlisting}[language=bash, caption={Endpoint Prometheus}]
GET /metrics
Content-Type: text/plain; version=0.0.4
\end{lstlisting}


% ============================================================================
% CHAPITRE 14 : KUBERNETES
% ============================================================================
\chapter{Kubernetes (Orchestration)}

\section{Deployment}

\begin{lstlisting}[language=yaml, caption={Extrait du Deployment Kubernetes}]
apiVersion: apps/v1
kind: Deployment
metadata:
  name: beewaf
spec:
  replicas: 1
  template:
    spec:
      containers:
        - name: beewaf
          image: beewaf:latest
          ports:
            - containerPort: 8000
          resources:
            requests: { cpu: "100m", memory: "128Mi" }
            limits: { cpu: "500m", memory: "512Mi" }
          livenessProbe:
            httpGet: { path: /health, port: 8000 }
            initialDelaySeconds: 15
            periodSeconds: 15
          readinessProbe:
            httpGet: { path: /health, port: 8000 }
            initialDelaySeconds: 5
            periodSeconds: 5
\end{lstlisting}

\section{Service}

\begin{lstlisting}[language=yaml, caption={Service Kubernetes}]
apiVersion: v1
kind: Service
metadata:
  name: beewaf-svc
spec:
  type: ClusterIP
  ports:
    - port: 80
      targetPort: 8000
\end{lstlisting}

\section{Ingress (TLS)}

\begin{lstlisting}[language=yaml, caption={Ingress Kubernetes avec TLS}]
apiVersion: networking.k8s.io/v1
kind: Ingress
metadata:
  name: beewaf-ingress
spec:
  ingressClassName: nginx
  tls:
    - hosts: ["beewaf.local"]
      secretName: beewaf-tls-secret
  rules:
    - host: beewaf.local
      http:
        paths:
          - path: /
            pathType: Prefix
            backend:
              service:
                name: beewaf-svc
                port: { number: 80 }
\end{lstlisting}


% ============================================================================
% CHAPITRE 15 : CI/CD — PIPELINE JENKINS
% ============================================================================
\chapter{CI/CD --- Pipeline Jenkins}

\section{Stages du Pipeline}

\begin{table}[H]
\centering
\caption{Stages du pipeline Jenkins}
\label{tab:jenkins}
\begin{tabularx}{\textwidth}{@{}clX@{}}
\toprule
\textbf{\#} & \textbf{Stage} & \textbf{Description} \\
\midrule
1 & Checkout & Récupération du code source (SCM) \\
2 & Install Dependencies & Création venv + \texttt{pip install -r requirements.txt} \\
3 & Run Unit Tests & Exécution \texttt{pytest -q} \\
4 & Build Docker Image & \texttt{docker build -t beewaf:\$\{BUILD\_NUMBER\}} \\
5 & Integration Test & Lancement conteneur + exécution test\_waf.sh \\
6 & Push Image & \texttt{docker tag + push} vers registry \\
7 & Deploy to K8s & \texttt{kubectl apply -f k8s/} \\
\bottomrule
\end{tabularx}
\end{table}

\section{Post-Actions}

\begin{itemize}[leftmargin=2em]
  \item \textbf{Always} : nettoyage conteneur Docker de test
  \item \textbf{Success} : notification de succès
  \item \textbf{Failure} : notification d'échec
\end{itemize}


% ============================================================================
% CHAPITRE 16 : CONFORMITÉ & COMPLIANCE
% ============================================================================
\chapter{Conformité \& Compliance (7 Frameworks)}

\section{Frameworks Supportés}

\begin{table}[H]
\centering
\caption{Frameworks de conformité supportés}
\label{tab:compliance-frameworks}
\begin{tabularx}{\textwidth}{@{}llX@{}}
\toprule
\textbf{Framework} & \textbf{Version} & \textbf{Couverture} \\
\midrule
OWASP Top 10 & 2021 & 100\% des 10 catégories couvertes par 10\,041 règles \\
PCI DSS & 4.0 & Requirement 6.4 (WAF), 6.5 (coding), 10.x (logging) \\
GDPR & 2018 & DLP (Art. 25, 32), Protection données personnelles \\
SOC 2 & Type II & Contrôles : CC6.1, CC6.6, CC6.7, CC7.2 \\
NIST & 800-53 Rev.5 & AC, AU, CM, IA, SC, SI controls \\
ISO 27001 & 2022 & A.8, A.12, A.14 \\
HIPAA & 2013 & § 164.312 (Access, Audit, Integrity, Transmission) \\
\bottomrule
\end{tabularx}
\end{table}

\section{OWASP Top 10 --- Mapping}

\begin{longtable}{@{}clL{9cm}@{}}
\caption{Couverture OWASP Top 10 par BeeWAF}
\label{tab:owasp-mapping}\\
\toprule
\textbf{\#} & \textbf{Catégorie OWASP} & \textbf{Couverture BeeWAF} \\
\midrule
\endfirsthead
\toprule
\textbf{\#} & \textbf{Catégorie OWASP} & \textbf{Couverture BeeWAF} \\
\midrule
\endhead
\bottomrule
\endfoot
A01 & Broken Access Control & Session Protection, JWT, CSRF, BOLA, IDOR detection \\
A02 & Cryptographic Failures & TLS 1.2/1.3, HSTS, Cookie security, DLP \\
A03 & Injection & SQLi (800+ rules), XSS (500+ rules), CMDi, LDAP, NoSQL, XPath \\
A04 & Insecure Design & Business logic checks, API security, Protocol validator \\
A05 & Security Misconfig. & Sensitive path blocking, Response cloaking, Header validation \\
A06 & Vulnerable Components & Virtual patching (37 CVE), Scanner detection \\
A07 & Auth Failures & Credential stuffing, Bot manager, Rate limiting, Brute force \\
A08 & Software/Data Integrity & Deserialization detection, File upload scanning \\
A09 & Logging \& Monitoring & ELK stack, Prometheus, JSON structured logging \\
A10 & SSRF & 200+ SSRF rules, Cloud metadata (AWS/GCP/Azure), DNS rebind \\
\end{longtable}


% ============================================================================
% CHAPITRE 17 : JEUX DE DONNÉES & ENTRAÎNEMENT ML
% ============================================================================
\chapter{Jeux de Données \& Entraînement ML}

\section{Script d'Entraînement}

\begin{lstlisting}[language=bash, caption={Commandes d'entraînement ML}]
# Entrainement complet
python train_ml_models.py --data data/csic_database.csv \
    --save models/ml_model.pkl

# Entrainement + evaluation
python train_ml_models.py --data data/csic_database.csv \
    --save models/ml_model.pkl --eval

# Evaluation uniquement (modele existant)
python train_ml_models.py --test-only

# Sortie JSON
python train_ml_models.py --test-only --json
\end{lstlisting}

\section{Métriques d'Entraînement Actuelles}

\begin{table}[H]
\centering
\caption{Métriques détaillées des modèles ML}
\label{tab:ml-metrics}
\begin{tabular}{@{}lccccc@{}}
\toprule
\textbf{Modèle} & \textbf{Accuracy} & \textbf{Precision} & \textbf{Recall} & \textbf{F1} & \textbf{ROC-AUC} \\
\midrule
IsolationForest & 77.3\% & 72.3\% & 72.6\% & 0.724 & --- \\
RandomForest & 94.2\% & 89.8\% & 96.7\% & 0.932 & 0.992 \\
GradientBoosting & 95.3\% & 94.5\% & 94.1\% & 0.943 & 0.993 \\
\bottomrule
\end{tabular}
\end{table}


% ============================================================================
% CHAPITRE 18 : TESTS & VALIDATION
% ============================================================================
\chapter{Tests \& Validation}

\section{Infrastructure de Tests}

\begin{table}[H]
\centering
\caption{Fichiers de tests}
\label{tab:test-files}
\begin{tabularx}{\textwidth}{@{}lllX@{}}
\toprule
\textbf{Fichier} & \textbf{Type} & \textbf{Framework} & \textbf{Couverture} \\
\midrule
run\_tests.py & Smoke test & FastAPI TestClient & Health, echo, basic SQLi/XSS \\
test\_admin\_rules.py & Unit test & pytest & Admin rules, ML-stats \\
test\_rate\_limit.py & Unit test & pytest & Rate limiting (65 requêtes) \\
test\_waf.sh & Integration & Bash/curl & Health, benign POST, SQLi, XSS \\
test\_all\_modules.py & \textbf{Complet} & requests & \textbf{39 sections, 261 tests} \\
quick\_ml\_test.py & ML quick test & Python & Validation ML prédictions \\
real\_time\_attacks.py & Stress test & Python & 10\,000+ attaques + FP \\
\bottomrule
\end{tabularx}
\end{table}

\section{Test Complet --- 39 Sections}

\begin{longtable}{@{}clcL{7.5cm}@{}}
\caption{Détail des 39 sections de test (261 tests)}
\label{tab:test-sections}\\
\toprule
\textbf{\#} & \textbf{Section} & \textbf{Tests} & \textbf{Couverture} \\
\midrule
\endfirsthead
\toprule
\textbf{\#} & \textbf{Section} & \textbf{Tests} & \textbf{Couverture} \\
\midrule
\endhead
\bottomrule
\endfoot
1 & Connectivité \& Info & 6 & Version, modules, health, ML, rules count \\
2 & Moteur Regex & 56 & 55 attaques (SQLi, XSS, CMDi, SSRF, XXE, LDAP, NoSQL, etc.) \\
3 & ML Engine & 6 & Stats, predict, classify \\
4 & Bot Detector & 7 & Normal UA, SQLMap, Nikto, Nmap, etc. \\
5 & Bot Manager Adv. & 3 & Credential stuffing, stats \\
6 & Rate Limiting & 3 & Normal, stats, config \\
7 & DDoS Protection & 3 & Normal, stats, seuils \\
8 & DLP & 3 & Credit card, SSN \\
9 & Geo/IP Blocking & 2 & Local IP, stats \\
10 & Protocol Validator & 4 & GET, invalid method, long URL, host injection \\
11 & API Security & 4 & JSON, nested, BOLA, GraphQL \\
12 & Threat Intelligence & 3 & Log4Shell, OAST, stats \\
13 & Session Protection & 3 & JWT alg:none, admin claim, stats \\
14 & Evasion Detector & 6 & URL-encoded, double, unicode, hex, mixed, null \\
15 & Correlation Engine & 3 & Endpoint, campaigns, events \\
16 & Adaptive Learning & 4 & Modes detect/enforce/learning, stats \\
17 & Response Cloaking & 8 & Server, X-Powered-By, X-Frame, etc. \\
18 & Cookie Security & 3 & Inspection, XSS, SQLi \\
19 & Virtual Patching & 5 & Endpoint, 37 patches, Log4Shell, Struts, Spring \\
20 & Zero-Day Detector & 3 & High entropy, binary chars, stats \\
21 & WebSocket Insp. & 2 & WS upgrade, malicious payload \\
22 & Payload Analyzer & 3 & PHP in GIF, polyglot, shell \\
23 & Compliance Engine & 9 & Endpoint, 7 frameworks \\
24 & API Discovery & 3 & Shadow API, GraphQL \\
25 & Threat Feed & 4 & MITRE ATT\&CK, C2, APT \\
26 & Cluster Manager & 3 & Stats, distributed, config \\
27 & Performance Engine & 10 & Response time, cache, bloom, dedup \\
28 & Sensitive Paths & 12 & .git, .env, wp-config, phpinfo, etc. \\
29 & Business Logic & 8 & XFF spoof, negative ID, IDOR, TE smuggling \\
30 & False Positives & 30 & 29 requêtes légitimes + compteur FP \\
31 & TLS/Nginx & 3 & HTTP$\to$HTTPS, HTTPS, HSTS \\
32 & Admin API & 10 & Auth reject, auth OK, wrong key \\
33 & Prometheus Metrics & 7 & Endpoint, 6 métriques \\
34 & Scanner Detection & 8 & SQLMap, Nikto, Nmap, Masscan, DirBuster, etc. \\
35 & File Upload & 3 & PHP webshell, JSP, double ext. \\
36 & Cloud Attacks & 4 & AWS, GCP, K8s, Docker \\
37 & Encoding Attacks & 4 & Unicode, overlong UTF-8, hex, double \\
38 & Windows Attacks & 3 & cmd.exe, PowerShell, UNC \\
39 & Performance Bench. & 5 & Avg, P95, P99, Max, détection \\
\midrule
\multicolumn{2}{l}{\textbf{TOTAL}} & \textbf{261} & \\
\end{longtable}

\section{Résultats du Test Complet (10 Février 2026)}

\begin{resultbox}[Résultats --- BeeWAF Enterprise v6.0]
\centering
\begin{tabular}{@{}lr@{}}
\textbf{Réussis} & 260 \\
\textbf{Échoués} & 0 \\
\textbf{Avertissements} & 1 (Empty UA -- cosmétique) \\
\textbf{Total} & 261 \\[0.5em]
\textbf{Taux de réussite} & \textbf{100.0\%} \\
\textbf{Grade fonctionnel} & \textbf{A+} \\[0.5em]
Attaques détectées & 55/55 (100\%) \\
Faux positifs & 0/29 (0\%) \\
Latence moyenne & 16ms \\
Latence P99 & 18ms \\
Temps détection attaque & 11ms \\
\end{tabular}
\end{resultbox}


% ============================================================================
% CHAPITRE 19 : RÉSULTATS DE PERFORMANCE
% ============================================================================
\chapter{Résultats de Performance}

\section{Benchmarks}

\begin{table}[H]
\centering
\caption{Résultats de performance}
\label{tab:benchmarks}
\begin{tabular}{@{}lcccc@{}}
\toprule
\textbf{Métrique} & \textbf{Valeur} & \textbf{Objectif} & \textbf{Statut} \\
\midrule
Latence moyenne & \textbf{16ms} & $\leq$50ms & \textcolor{beegreen}{\checkmark} \\
Latence P95 & \textbf{18ms} & $\leq$50ms & \textcolor{beegreen}{\checkmark} \\
Latence P99 & \textbf{18ms} & $\leq$100ms & \textcolor{beegreen}{\checkmark} \\
Latence max & \textbf{18ms} & $\leq$200ms & \textcolor{beegreen}{\checkmark} \\
Temps détection attaque & \textbf{11ms} & $\leq$20ms & \textcolor{beegreen}{\checkmark} \\
Démarrage à froid & \textbf{$\sim$12s} & $\leq$15s & \textcolor{beegreen}{\checkmark} \\
Compilation 10\,041 règles & \textbf{$<$5s} & $\leq$10s & \textcolor{beegreen}{\checkmark} \\
Taux de détection & \textbf{98.2\%} & $\geq$95\% & \textcolor{beegreen}{\checkmark} \\
Taux de faux positifs & \textbf{0\%} & $\leq$2\% & \textcolor{beegreen}{\checkmark} \\
\bottomrule
\end{tabular}
\end{table}

\section{Comparaison avec Solutions Commerciales}

\begin{table}[H]
\centering
\caption{Comparaison étendue avec solutions commerciales}
\label{tab:comparison-extended}
\small
\begin{tabular}{@{}lccccc@{}}
\toprule
\textbf{Métrique} & \textbf{BeeWAF} & \textbf{F5 BIG-IP} & \textbf{ModSecurity} & \textbf{AWS WAF} & \textbf{Cloudflare} \\
\midrule
Score & \textbf{98.2/100} & 73/100 & 65/100 & $\sim$70/100 & $\sim$80/100 \\
Grade & \textbf{A+} & B & C+ & B- & B+ \\
Règles & \textbf{10\,041} & $\sim$2\,500 & $\sim$900 & $\sim$200 & $\sim$5\,000 \\
ML & \textbf{3 modèles} & Limité & Non & Limité & Oui \\
FP Rate & \textbf{0\%} & $\sim$5\% & $\sim$8\% & $\sim$3\% & $\sim$2\% \\
Latence & \textbf{16ms} & $\sim$5ms & $\sim$20ms & $\sim$2ms & $\sim$1ms \\
Open Source & \textbf{Oui} & Non & Oui & Non & Non \\
\bottomrule
\end{tabular}
\end{table}


% ============================================================================
% CHAPITRE 20 : DÉPENDANCES & PRÉREQUIS
% ============================================================================
\chapter{Dépendances \& Prérequis}

\section{Prérequis Système}

\begin{table}[H]
\centering
\caption{Prérequis système}
\label{tab:prerequisites}
\begin{tabular}{@{}lcc@{}}
\toprule
\textbf{Composant} & \textbf{Version Minimum} & \textbf{Recommandé} \\
\midrule
Docker & 20.x & 24+ \\
Docker Compose & 2.x & 2.24+ \\
Python & 3.11 & 3.11 \\
RAM & 2 Go & 4 Go+ (avec ELK) \\
Disque & 5 Go & 20 Go+ (avec logs) \\
CPU & 2 c\oe{}urs & 4 c\oe{}urs \\
\bottomrule
\end{tabular}
\end{table}

\section{Ports Réseau}

\begin{table}[H]
\centering
\caption{Ports réseau utilisés}
\label{tab:ports}
\begin{tabular}{@{}clc@{}}
\toprule
\textbf{Port} & \textbf{Service} & \textbf{Protocole} \\
\midrule
80 & Nginx HTTP (redirect) & TCP \\
443 & Nginx HTTPS & TCP \\
8000 & BeeWAF FastAPI (interne) & TCP \\
9200 & Elasticsearch & TCP \\
5044 & Logstash (beats) & TCP/UDP \\
5601 & Kibana & TCP \\
9600 & Logstash monitoring & TCP \\
\bottomrule
\end{tabular}
\end{table}


% ============================================================================
% CHAPITRE 21 : VARIABLES D'ENVIRONNEMENT
% ============================================================================
\chapter{Variables d'Environnement}

\begin{table}[H]
\centering
\caption{Variables d'environnement}
\label{tab:env-vars}
\small
\begin{tabularx}{\textwidth}{@{}llXc@{}}
\toprule
\textbf{Variable} & \textbf{Défaut} & \textbf{Description} & \textbf{Oblig.} \\
\midrule
\texttt{BEEWAF\_API\_KEY} & \texttt{changeme-*} & Clé API administration & $\triangle$ \\
\texttt{BEEWAF\_MODEL\_PATH} & models/anomaly*.pkl & Chemin modèle anomaly legacy & Non \\
\texttt{BEEWAF\_ML\_ENGINE\_PATH} & models/ml\_model.pkl & Chemin modèle ML ensemble & Non \\
\texttt{BEEWAF\_TRAIN\_DATA} & train\_demo.csv & Données entraînement legacy & Non \\
\texttt{BEEWAF\_CSIC\_DATA} & csic\_database.csv & Dataset CSIC pour ML & Non \\
\texttt{BEEWAF\_ML\_MODE} & advanced & Mode ML : legacy / advanced & Non \\
\texttt{BEEWAF\_ALLOWED\_HOSTS} & \emph{(vide)} & Hosts autorisés (comma-separated) & Non \\
\texttt{BEEWAF\_RULES\_FILE} & \emph{(vide)} & Fichier de règles supplémentaires & Non \\
\bottomrule
\end{tabularx}
\end{table}

\begin{warnbox}[Attention]
La variable \texttt{BEEWAF\_API\_KEY} doit impérativement être changée en production. La valeur par défaut \texttt{changeme-default-key-not-secure} ne doit jamais être utilisée en environnement de production.
\end{warnbox}


% ============================================================================
% CHAPITRE 22 : SÉCURITÉ & AUTHENTIFICATION
% ============================================================================
\chapter{Sécurité \& Authentification}

\section{TLS/SSL}

\begin{itemize}[leftmargin=2em]
  \item \textbf{Protocoles} : TLS 1.2 et TLS 1.3 uniquement
  \item \textbf{Chiffrement} : Suites ECDHE (Perfect Forward Secrecy)
  \item \textbf{HSTS} : \texttt{max-age=31536000; includeSubDomains}
  \item \textbf{Certificats} : \texttt{/etc/nginx/ssl/tls.crt} + \texttt{/etc/nginx/ssl/tls.key}
\end{itemize}

\section{Headers de Sécurité}

\begin{table}[H]
\centering
\caption{Headers de sécurité complets}
\label{tab:security-headers}
\begin{tabularx}{\textwidth}{@{}llX@{}}
\toprule
\textbf{Header} & \textbf{Valeur} & \textbf{Protection} \\
\midrule
\texttt{X-Frame-Options} & \texttt{DENY} & Anti-clickjacking \\
\texttt{X-Content-Type-Options} & \texttt{nosniff} & Anti-MIME sniffing \\
\texttt{X-XSS-Protection} & \texttt{1; mode=block} & Filtre XSS navigateur \\
\texttt{Strict-Transport-Security} & \texttt{max-age=31536000} & Force HTTPS \\
\texttt{Referrer-Policy} & \texttt{strict-origin-*} & Contrôle Referer \\
\texttt{Permissions-Policy} & \texttt{geolocation=(), ...} & Restrictions API \\
\texttt{Server} & \emph{(masqué)} & Cloaking serveur \\
\texttt{X-Powered-By} & \emph{(supprimé)} & Cloaking technologie \\
\bottomrule
\end{tabularx}
\end{table}

\section{Authentification API Admin}

\begin{lstlisting}[language=bash, caption={Authentification API}]
Header requis : X-API-Key
Cle configuree via : BEEWAF_API_KEY
Reponses : 401 (absente), 403 (invalide), 200 (valide)
\end{lstlisting}

\section{Protection contre les Abus}

\begin{itemize}[leftmargin=2em]
  \item \textbf{Rate Limiting} : Configurable par méthode HTTP (GET/POST)
  \item \textbf{IP Blocklist} : Blocage automatique après seuil de violations
  \item \textbf{DDoS Protection} : 3 niveaux (warn/throttle/block)
  \item \textbf{Credential Stuffing} : Détection login rapide ($>$5 tentatives/60s)
\end{itemize}


% ============================================================================
% CHAPITRE 23 : ÉVOLUTIONS & HISTORIQUE
% ============================================================================
\chapter{Évolutions \& Historique des Versions}

\section{Changelog}

\begin{table}[H]
\centering
\caption{Historique des versions}
\label{tab:changelog}
\begin{tabularx}{\textwidth}{@{}ccX@{}}
\toprule
\textbf{Version} & \textbf{Date} & \textbf{Changements Majeurs} \\
\midrule
v1.0 & 2025 & WAF basique : règles regex, anomaly detector IsolationForest \\
v2.0 & 2025 & Ajout rate limiting, bot detection, Docker Compose \\
v3.0 & 2025 & Stack ELK (Elasticsearch + Logstash + Kibana + Filebeat) \\
v4.0 & 2025 & 15 modules avancés, 425+ règles, score 82.5 (bat F5 BIG-IP : 73) \\
v5.0 & Jan 2026 & 27 modules, ML ensemble 3 modèles, 7 frameworks compliance, score 98.3/100 Grade A+ \\
\textbf{v6.0} & \textbf{Fév 2026} & \textbf{10\,041 règles, 0\% FP, 37 CVE patches, 100\% tests (260/260)} \\
\bottomrule
\end{tabularx}
\end{table}

\section{Métriques d'Évolution}

\begin{figure}[H]
\centering
\begin{tikzpicture}
  \draw[thick, -Stealth] (0,0) -- (12,0) node[right] {Version};
  \draw[thick, -Stealth] (0,0) -- (0,6) node[above] {Score /100};

  % Grid
  \foreach \y in {20,40,60,80,100} {
    \draw[gray!30] (0,\y/20) -- (11.5,\y/20);
    \node[left, font=\tiny] at (0,\y/20) {\y};
  }

  % Data points
  \fill[beered] (1,3) circle (4pt) node[above, font=\scriptsize] {v1.0: 60};
  \fill[beered!80] (3,3.4) circle (4pt) node[above, font=\scriptsize] {v2.0: 68};
  \fill[beeyellow] (5,3.6) circle (4pt) node[above, font=\scriptsize] {v3.0: 72};
  \fill[beeblue] (7,4.125) circle (4pt) node[above, font=\scriptsize] {v4.0: 82.5};
  \fill[beegreen] (9,4.915) circle (4pt) node[above, font=\scriptsize] {v5.0: 98.3};
  \fill[beegreen!80!black] (11,4.91) circle (5pt) node[above, font=\scriptsize\bfseries] {v6.0: 98.2};

  % Line
  \draw[thick, beedark] (1,3) -- (3,3.4) -- (5,3.6) -- (7,4.125) -- (9,4.915) -- (11,4.91);
\end{tikzpicture}
\caption{Évolution du score de détection par version}
\label{fig:evolution}
\end{figure}


% ============================================================================
% CHAPITRE 24 : ANNEXES
% ============================================================================
\chapter{Annexes}

\section{Structure du Projet}

\begin{lstlisting}[language=bash, caption={Arborescence complète du projet}, basicstyle=\ttfamily\scriptsize]
beehivepfe2-main/
  app/
    main.py                    # Application FastAPI (~1 317 lignes)
  waf/
    __init__.py                # Package WAF (27 modules)
    rules.py                   # Moteur de regles principal
    rules_extended.py          # 586 regles etendues
    rules_advanced.py          # 425 regles avancees v4.0
    rules_v5.py                # 1 207 regles v5.0
    rules_mega_1.py            # 1 120 regles mega pack 1
    rules_mega_2.py            # 542 regles mega pack 2
    rules_mega_3.py            # 412 regles mega pack 3
    rules_mega_4.py            # 292 regles mega pack 4
    rules_mega_5.py            # 313 regles mega pack 5
    rules_mega_6.py            # 214 regles mega pack 6
    rules_mega_7.py            # 1 091 regles mega pack 7
    rules_mega_8.py            # 1 161 regles mega pack 8
    rules_mega_9.py            # 887 regles mega pack 9
    rules_mega_10.py           # 776 regles mega pack 10
    rules_mega_11.py           # 576 regles mega pack 11
    rules_mega_12.py           # 152 regles mega pack 12
    ml_engine.py               # Moteur ML ensemble 3 modeles
    anomaly.py                 # Anomaly detector legacy
    ratelimit.py               # Rate limiter + IP blocklist
    bot_detector.py            # Detection bots
    bot_manager_advanced.py    # Bot manager avance
    dlp.py                     # Data Loss Prevention
    geo_block.py               # Blocage geographique
    protocol_validator.py      # Validation protocole HTTP
    api_security.py            # Securite API
    threat_intel.py            # Threat Intelligence
    threat_feed.py             # Threat Feed
    session_protection.py      # Protection session
    evasion_detector.py        # Detecteur d'evasion
    correlation_engine.py      # Moteur de correlation
    adaptive_learning.py       # Apprentissage adaptatif
    response_cloaking.py       # Camouflage reponse
    cookie_security.py         # Securite cookies
    virtual_patching.py        # Patches virtuels CVE
    zero_day_detector.py       # Detecteur zero-day
    websocket_inspector.py     # Inspecteur WebSocket
    payload_analyzer.py        # Analyseur payload
    compliance_engine.py       # Moteur conformite
    ddos_protection.py         # Protection DDoS
    api_discovery.py           # Decouverte API
    cluster_manager.py         # Manager cluster
    performance_engine.py      # Moteur performance
    clamav_scanner.py          # Scanner ClamAV (optionnel)
  data/
    csic_database.csv          # Dataset CSIC 2010 (61 065 samples)
    train_demo.csv             # Dataset demo
    train_kaggle.csv           # Dataset Kaggle
    train_synthetic.csv        # Dataset synthetique
  models/
    anomaly_model.pkl          # Modele legacy
    ml_model.pkl               # Modele ML ensemble
  elk/
    filebeat/filebeat.yml      # Config Filebeat
    logstash/
      config/logstash.yml      # Config Logstash
      pipeline/beewaf.conf     # Pipeline Logstash
  k8s/
    deployment.yaml            # Deploiement K8s
    service.yaml               # Service K8s
    ingress.yaml               # Ingress TLS K8s
    tls-secret.yaml            # Secret TLS
    tls/                       # Certificats TLS
  tests/
    run_tests.py               # Smoke tests
    test_admin_rules.py        # Tests admin
    test_rate_limit.py         # Tests rate limit
    test_waf.sh                # Tests integration bash
  docker-compose-elk.yaml      # Docker Compose (6 services)
  docker-compose.yaml          # Docker Compose simple
  Dockerfile.full              # Dockerfile complet (principal)
  nginx.conf                   # Configuration Nginx
  Jenkinsfile                  # Pipeline CI/CD Jenkins
  requirements.txt             # Dependances Python
  train_ml_models.py           # Script entrainement ML
  CAHIER_DE_CHARGE.tex         # Ce document (LaTeX)
\end{lstlisting}

\section{Glossaire}

\begin{longtable}{@{}lL{10cm}@{}}
\caption{Glossaire des termes techniques}
\label{tab:glossaire}\\
\toprule
\textbf{Terme} & \textbf{Définition} \\
\midrule
\endfirsthead
\toprule
\textbf{Terme} & \textbf{Définition} \\
\midrule
\endhead
\bottomrule
\endfoot
WAF & Web Application Firewall --- pare-feu applicatif web \\
ML & Machine Learning --- apprentissage automatique \\
FP & False Positive --- faux positif (requête légitime bloquée à tort) \\
OWASP & Open Web Application Security Project \\
PCI DSS & Payment Card Industry Data Security Standard \\
GDPR & General Data Protection Regulation \\
HIPAA & Health Insurance Portability and Accountability Act \\
SOC 2 & Service Organization Control Type 2 \\
NIST & National Institute of Standards and Technology \\
ELK & Elasticsearch + Logstash + Kibana \\
SSRF & Server-Side Request Forgery \\
XSS & Cross-Site Scripting \\
SQLi & SQL Injection \\
CMDi & Command Injection \\
XXE & XML External Entity \\
SSTI & Server-Side Template Injection \\
CSRF & Cross-Site Request Forgery \\
BOLA & Broken Object Level Authorization \\
IDOR & Insecure Direct Object Reference \\
DLP & Data Loss Prevention \\
DDoS & Distributed Denial of Service \\
CVE & Common Vulnerabilities and Exposures \\
MITRE ATT\&CK & Framework de classification des techniques d'attaque \\
JA3 & TLS fingerprinting method \\
HSTS & HTTP Strict Transport Security \\
\end{longtable}

\section{Références}

\begin{enumerate}[leftmargin=2em]
  \item OWASP Top 10 (2021) --- \url{https://owasp.org/Top10/}
  \item CSIC 2010 HTTP Dataset --- Universidad Carlos III de Madrid
  \item PCI DSS v4.0 --- \url{https://www.pcisecuritystandards.org/}
  \item NIST 800-53 Rev.5 --- \url{https://csrc.nist.gov/publications/detail/sp/800-53/rev-5/final}
  \item MITRE ATT\&CK --- \url{https://attack.mitre.org/}
  \item scikit-learn Documentation --- \url{https://scikit-learn.org/}
  \item FastAPI Documentation --- \url{https://fastapi.tiangolo.com/}
\end{enumerate}

\vfill

\begin{center}
\rule{0.8\textwidth}{0.4pt}\\[0.5em]
{\small\bfseries Document généré le 10 Février 2026}\\[0.3em]
{\small BeeWAF Enterprise v6.0 --- 10\,041 règles | 3 modèles ML | 27 modules | 7 frameworks}\\[0.3em]
{\small\color{beegreen}\bfseries Grade Fonctionnel : A+ (260/260 tests, 100\%)}
\end{center}

\end{document}
